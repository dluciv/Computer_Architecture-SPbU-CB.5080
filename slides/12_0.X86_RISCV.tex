\documentclass[xetex,aspectratio=43]{beamer}

\usepackage{res/lections}

\preamble

\title{Примеры архитектур: Intel x86 и RISC-V}

\begin{document}

    \titleslide

    \tocslide


%==============================================

\section{Архитектура и система команд x86}

\subsection{Регистры и адресация}

\begin{frame}{Регистровый файл (1)}
        Регистры (для 32-битной ЭВМ)

        \begin{itemize}
            \tightlist
            \item
            EAX (общий, аккумулятор), EDX (умножение и деление вместе с EAX), EBX
            (указатели), ECX (счетчик)
            \item
            EDI (dest index), ESI (source index)
            \item
            EBP, ESP, EIP
            \item
            CS, SS, DS, ES, FS, GS --- сегментные
            \item
            EFLAGS
        \end{itemize}

        \pause

        Фрагменты регистров

        \begin{itemize}
            \tightlist
            \item
            \_H, \_L --- 8-разрядные
            \item
            \_X, \_S --- 16-разрядные
            \item
            E\_ --- 32-разрядные
            \item
            R\_ --- 64-разрядные
        \end{itemize}

        Например, для аккумулятора

        (AH (8), AL (8)) $\rightarrow$ AX (16) $\rightarrow$ EAX (32) $\rightarrow$ RAX (64)
\end{frame}

\begin{frame}{Регистровый файл (2): EFLAGS}
Регистр флагов EFLAGS.
    Весь регистр 32-битный (начиная с 80386).

    Основные флаги (с 8086):

    \begin{itemize}
        \tightlist
        \item
        OF --- флаг переполнения
        \item
        DF --- флаг направления
        \item
        IF --- флаг прерывания
        \item
        TF --- флаг трассировки
        \item
        SF --- флаг знака
        \item
        ZF --- флаг нуля
        \item
        AF --- флаг дополнительного переноса (для упакованных
        двоично-десятичных операций)
        \item
        PF --- флаг четности
        \item
        CF --- флаг переноса
    \end{itemize}
\end{frame}

\begin{frame}{Адресация данных}
    \begin{itemize}
        \item
        Непосредственная (аргументы в коде)
        \item
        Регистровая (номер регистра в коде)
        \item
        Память{[}E\_X + смещение{]}, Память{[}EBP + смещение{]}, + возможно
        префиксы сегментов
    \end{itemize}
\end{frame}

\subsection{Система команд}

\begin{frame}{Команды пересылки данных}
    \begin{itemize}
        \tightlist
        \item
        MOV память обменивается только с арифметическими регистрами, ESI, EDI
        \item
        XCHG reg, mem/reg
        \item
        LAHF, SAHF --- флаги \(\leftrightarrow\) AH
    \end{itemize}
\end{frame}

\begin{frame}{Команды АЛУ}
    \begin{block}{Логические}
        AND, OR, XOR, NOT
    \end{block}

    \begin{block}{Арифметические}
        \begin{itemize}
            \item
            ADD, SUB, ADC, SBB, INC, DEC, NEG
            \item
            MUL (reg/mem), DIV (reg/mem), IMUL, IDIV,
            \item
            CWQ (EAX \(\rightarrow\) EDX:EAX)
        \end{itemize}
    \end{block}

    \begin{block}{Сдвига}
        \begin{itemize}
            \item
            ROR, ROL
            \item
            RCL, RCR --- с переносом
            \item
            SHL, SHR --- без переноса
            \item
            SAL, SAR --- со знаковыми битами
        \end{itemize}
    \end{block}
\end{frame}

\begin{frame}{BCD}
    ASCII и BCD --- для быстрого преобразования двоично-десятичных чисел
\end{frame}

\begin{frame}{Команды работы со стеком}
    \begin{itemize}
        \item
        PUSH, POP
        \item
        PUSHA, POPA
        \item
        Косвенно --- CALL, RET, INT, IRET
    \end{itemize}
\end{frame}

\begin{frame}{Команды сравнения и передачи управления}
    \begin{block}{Переходы Безусловные}
        \begin{itemize}
            \tightlist
            \item
            JMP FAR, NEAR, JMP M{[}xx{]}, JMP REG
        \end{itemize}
    \end{block}

    \begin{block}{Команды Сравнения}
        \begin{itemize}
            \item
            CMP --- как SUB
            \item
            TEST --- как AND
            \item
            CMPS --- CMPSB, CMPSW, CMPSD
        \end{itemize}

        \pause

        Compare-exchange
        \begin{itemize}
            \tightlist
            \item
            CMPXCHG dest, src --- Сравнивает аккумулятор (8-64 bits) с dest. Если
            равны, то в dest грузят src, иначе в аккумулятор загружают dest
        \end{itemize}

        \pause

        Ужас. Кошмар.
        \href{https://ru.wikipedia.org/wiki/\%D0\%A1\%D1\%80\%D0\%B0\%D0\%B2\%D0\%BD\%D0\%B5\%D0\%BD\%D0\%B8\%D0\%B5_\%D1\%81_\%D0\%BE\%D0\%B1\%D0\%BC\%D0\%B5\%D0\%BD\%D0\%BE\%D0\%BC\#\%D0\%97\%D0\%B0\%D1\%87\%D0\%B5\%D0\%BC_\%D1\%8D\%D1\%82\%D0\%BE_\%D0\%BD\%D1\%83\%D0\%B6\%D0\%BD\%D0\%BE}{Для чего она?..}
    \end{block}
\end{frame}

\begin{frame}{Условные переходы I}
    По результату \(R\) или итогам сравнения \(A ? B\), в зависимости от
    получившихся значений флагов.

    Беззнаковые

    \begin{itemize}
        \tightlist
        \item
        JA/JNBE --- если \(A > B\);
        \item
        JAE/JNB/JNC --- если \(A \ge B\);
        \item
        JB/JNAE/JC --- если \(A < B\);
        \item
        JBE/JNA --- если \(A \le B\).
    \end{itemize}

    Знаковые

    \begin{itemize}
        \tightlist
        \item
        JG/JNLE --- если \(A > B\);
        \item
        JGE/JNL --- если \(A \ge B\);
        \item
        JL/JNGE --- если \(A < B\);
        \item
        JLE/JNG --- если \(A \le B\);
        \item
        JNS --- если \(R\ge 0\);
        \item
        JS --- если \(R<0\).
    \end{itemize}
\end{frame}

\begin{frame}[fragile]{Условные переходы II}
    По результату \(R\) или итогам сравнения \(A ? B\), в зависимости от
    получившихся значений флагов.

    \begin{itemize}
        \item
        JE/JZ --- если \(A=B \lor R = 0\);
        \item
        JNE/JNZ --- если \(A \not = B \lor R \not = 0\);
        \item
        JNO --- \(\neg OF\);
        \item
        JO --- \(OF\);
        \item
        JCXZ --- \(CX = 0\) --- для организации циклов\\
        \texttt{do\ ...\ while(-\/-CX);};
        \item
        JNP/JPO --- \(\neg PF\);
        \item
        JP/JPE --- \(PF\).
    \end{itemize}
\end{frame}

\begin{frame}{Вызовы и прерывания}
    \begin{itemize}
        \tightlist
        \item
        Вызовы

        \begin{itemize}
            \tightlist
            \item
            CALL адрес
        \end{itemize}
        \item
        Прерывания

        \begin{itemize}
            \tightlist
            \item
            Управление STI, CLI
            \item
            Ожидание (HALT)
        \end{itemize}
    \end{itemize}
\end{frame}

\begin{frame}{Команды ввода-вывода}
    IN (mem/DX), OUT (mem/DX) --- с AL
\end{frame}

\begin{frame}{Команды обработки строк (микроциклы)}
    \begin{itemize}
        \item
        REP, REPE, REPZ, REPNE, REPNZ
        \item
        LODS (загружает в аккумулятор),\\
        STOS (пишет из аккумулятора),\\
        MOVS (B-W-D --- пересылка память-память),\\
        CMPS(сравнение память-память),\\
        SCAS (вычитает из аккумулятора)
    \end{itemize}

    \pause

    Команды учитывают DF --- флаг направления. Выставив его «неправильно»
    можно быстро размножить участок памяти
\end{frame}

\begin{frame}{Команды математического сопроцессора и математического блока}
    \begin{block}{Стековые}
        \begin{itemize}
        \tightlist
        \item
        FLD, FSTP --- загрузить из памяти / выгрузить в память, формат
        \item
        FILD, FLD, ... --- загрузить из регистра целое / выгрузить в регистр целое
        \item
        FMUL, ... --- операции и функции
        \item
        FWAIT --- соинхронизация для старых процессоров
        \end{itemize}
    \end{block}
    \begin{block}{Регистровые}
        MULSD, MULSF --- работают с векторными регистрами (появились в Pentium MMX, позволяют производить по несколько операций с числами разной длины)
    \end{block}
\end{frame}

\begin{frame}{«Привелегированные» команды}
        Загрузка таблиц страниц и дескрипторов, изменение режимов работы
        процессора
\end{frame}

%----------------------------------------------

\section{Архитектура и система команд RISC-V}

\subsection{Регистры}

\subsection{Соглашение о вызовах}

\subsection{Команды и формат машинного кода}

\subsection{Кросс-компиляция}

\begin{frame}{Кросс-компиляция}
    \begin{itemize}
        \item При помощи \href{https://buildroot.org/}{BuildRoot}
        \item С отладкой — при помощи \href{http://ripes.me/Ripes/}{Ripes} %или \href{http://ripes.me/}{Онлайнового} Ripes
    \end{itemize}
\end{frame}

%==============================================

\section*{}

\begin{frame}{Вопросы и упражнения}
    \begin{block}{Вопросы}
        \protect\hypertarget{ux432ux43eux43fux440ux43eux441ux44b}{}
        \begin{enumerate}
            \tightlist
            \item
            Приведите примеры арифметико-логических команд x86
            \item
            Что такое микроциклы?
            \item
            Приведите примеры и опишите работу нескольких команд условного
            перехода
            \item
            Что такое лексема?
            \item
            Что такое синтаксическое дерево?
            \item
            Назовите обязательный стадии трансляции
            \item
            Назовите опциональные стадии трансляции
        \end{enumerate}
    \end{block}

    \begin{block}{Упражнения}
        \protect\hypertarget{ux443ux43fux440ux430ux436ux43dux435ux43dux438ux44f}{}
        \begin{enumerate}
            \tightlist
            \item
            Попробуйте воспользоваться претрансляцией для любого языка (Scheme,
            Julia, Nemerle, \ldots)
            \item
            Скомпилируйте программу из примера для любой незнакомой архитектуры;
            пользуясь справочниками, объясните действия всех машинных команд
        \end{enumerate}
    \end{block}
\end{frame}

\postamble

\end{document}
