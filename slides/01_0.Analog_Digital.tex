\documentclass[xetex,aspectratio=43]{beamer}

\usepackage{res/lections}

\preamble

\title[Аналоговые и цифровые ВС]{Аналоговые и цифровые вычислительные системы}

\begin{document}

\titleslide

\tocslide

\section{Аналоговые и цифровые устройства}
	
\begin{frame}{Аналоговый и цифровой сигналы}

	\href{http://protect.gost.ru/v.aspx?control=7\&id=152608}{ГОСТ 17657-79~\extlink}:
	
	\defn{Аналоговый сигнал}{сигнал данных, у которого каждый из
	представляющих параметров описывается функцией \emph{времени} и
	\emph{непрерывным множеством} возможных значений}
	
	\defn{Цифровой сигнал}{сигнал данных, у которого каждый из
	представляющих параметров описывается функцией \emph{дискретного
		времени} и \emph{конечным} множеством возможных значений}
	
	\pause
	
	Почему аналоговый?
	
	\begin{itemize}
		\item Для измерения требует сопоставления с эталоном
	\end{itemize}
\end{frame}

\section{Аналоговые устройства}

\begin{frame}{Аналоговые вычислительные системы}
	\defn{Аналоговое устройство}{устройство, представляющее данные в
	виде аналоговых сигналов}
	
	\pause
	
	По материалам из Большой Советской Энциклопедии
	
	\begin{itemize}
		
		\item
		Др. Греция ---
		\href{https://ru.wikipedia.org/wiki/\%D0\%9F\%D0\%B0\%D0\%BD\%D1\%82\%D0\%BE\%D0\%B3\%D1\%80\%D0\%B0\%D1\%84_(\%D0\%BF\%D1\%80\%D0\%B8\%D0\%B1\%D0\%BE\%D1\%80)\#\%D0\%98\%D0\%B7\%D0\%BE\%D0\%B1\%D1\%80\%D0\%B0\%D0\%B6\%D0\%B5\%D0\%BD\%D0\%B8\%D1\%8F}{пантограф~\extlink}
		\item
		Около 1600 г. --- логарифмическая линейка
		\item
		Около 1800 г. --- сложные
		\href{https://en.wikipedia.org/wiki/Nomogram}{номограммы~\extlink}, например,
		для навигации --- позволяют вычислять функции от многих переменных
		(температура смесей, площади стандартных фигур)
		\item
		В 1814 (Дж. Герман) --- планиметр, ранее --- курвиметры.
		\href{https://youtu.be/kwSN7edZZnM}{Просты в изготовлении~\extlink}
		\item 
		\item
		1940-е годы --- операционные усилители, сначала на лампах, потом на
		полупроводниках.
		\item
		1970-е постепенный спад
	\end{itemize}
\end{frame}

\begin{frame}{Преимущества аналоговых устройств}
	Язык природы
	
	\begin{itemize}
		
		\item
		Логарифмические рецепторные кривые: \(\ln'_{x}x = \frac{1}{x}\), чем
		больше абсолютное значение, тем ниже точность (подробнее на курсе по
		алгоритмам и структурам данных)
		\item
		Компактность решения конкретной задачи
		\item
		«Непрерывное» представление информации
	\end{itemize}
	
\end{frame}

\begin{frame}{Недостатки аналоговых устройств}
	\begin{itemize}
		
		\item
		Неуниверсальность
		\item
		Субъективность
		\item
		Проблемы преобразования:
		
		\begin{itemize}
			
			\item
			коэффициенты нелинейности трактов (ряд Тейлора)
			\item
			промежутки монотонности
		\end{itemize}
	\end{itemize}
\end{frame}

\begin{frame}{Нелинейность}
	Пусть:
	
	\[y = f(x) = c_{f0} + c_{f1} x + c_{f2} x^2 + R_f(x),\]
	\[z = g(y) = c_{g0} + c_{g1} y + c_{g2} y^2 + R_g(y)\]
	
	Тогда:
	
	\[ z = (g\cdot f)(x) = c_{g0} + \dots c_{f2} c_{g2} x^4 + \dots \]
	
	\begin{center}
		\includesvg[width=0.5\textwidth]{img/01.non-linear.svg}
	\end{center}
\end{frame}

\section{Цифровые устройства}

\begin{frame}{Цифровые устройства}
	\defn{Цифровое устройство}{устройство, представляющее данные в
		виде цифровых сигналов}
\end{frame}

\begin{frame}{Недостатки цифровых устройств}
	\begin{itemize}
		
		\item
		По началу (иногда --- до сих пор) громоздкое оборудование
		\item
		Символы вместо естественных значений
	\end{itemize}
\end{frame}
	
\begin{frame}{Преимущества цифровых устройств}
	\begin{itemize}
		
		\item
		Универсальность
		\item
		С конца 50-х --- программируемость (Фортран, Кобол, Алгол, Лисп)
		\item
		Модульность, откуда:
		
		\begin{itemize}
			
			\item
			легкая сопрягаемость
			\item
			легко проектировать
		\end{itemize}
	\end{itemize}
\end{frame}

\section{Модуляция и передача аналоговых сигналов}

\begin{frame}{Подопытная функция}
	\[f(x) := \frac{sin(x) +cos(2x) +2}{2}\]

	\begin{center}
		\includesvg[height=0.75\textheight]{img/01.mod-fun.svg}
	\end{center}
\end{frame}

\begin{frame}{Несущая и модуляция}
	Не любой сигнал можно передать в исходном виде. Например, радиопередача
	на частотах, типичных для голоса, технически сложна. Низкочастотный
	сигнал передают при помощи высокочастотной радиопередачи
	
	\pause
	
	\defn{Несущая}{периодическая функция, искажение которой
	используется для передачи сигнала}
	
	\defn{Модуляция}{способ искажения несущей для передачи сигнала}
\end{frame}

\begin{frame}{Амплитудная модуляция}
	\[s(x) = f(x)\sin(10 x)\]
	
	\begin{center}
		\includesvg[height=0.75\textheight]{img/01.mod-amp.svg}
	\end{center}
	
	Здесь несущая --- \(sin(10 x)\)
\end{frame}

\begin{frame}{Фазовая модуляция}
	\[s(x) = sin(10x+5f(x))\]

	\begin{center}
		\includesvg[height=0.75\textheight]{img/01.mod-pha.svg}
	\end{center}	

	
	Здесь несущая --- \(sin(10 x)\)
\end{frame}

\begin{frame}{Частотная модуляция}
	\[s(x) = sin\left(5x+5\left(\int_x f\left( x\right) dx \right)\right)\]
	
	\begin{center}
		\includesvg[height=0.75\textheight]{img/01.mod-fre.svg}
	\end{center}	
	
	Здесь несущая --- \(sin(5 x)\)
\end{frame}

\begin{frame}{Теорема Котельникова}
	Если мы передаём при помощи несущего сигнала периодическую функцию,
	максимальная частота в спектре которой --- \(P\), то несущая должна
	иметь частоту \(F\), такую, что
	
	\[F \ge 2P\]
	
	Без доказательства		
\end{frame}

\section{Передача цифровых сигналов}

\begin{frame}{Символьное пространство, шум}
	Число \(N\) в системе счисления с основанием \(b\) записывается
	приблизительно
	
	\[\log_b N\]
	
	цифрами.
	
	\pause
	
	\(M\) --- емкость символьного пространства. Передаём \(x\) битов. В
	символах это будет \(log_M (2^x)\).
	
	Передаём 1 символ.
	\[1 = \log_M (2^x) = \frac{\log_2 (2^x)}{\log_2 M} = \frac{x}{\log_2 M}\]
	\[x = \log_2 M\]
\end{frame}

\begin{frame}{Скорость передачи}
	\(P\) бод --- скорость передачи данных. За раз передаём 1 из \(M\)
	уровней. Тогда: \[V = P \log_2 M\]
\end{frame}

\begin{frame}{Теорема Шеннона}
	Мощность алфавита \[M \le 1+S/N,\]
	
	где \(S\) --- мощность сигнала, \(N\) --- мощность шума
	
	Без доказательства
	
	\pause
	
	Таким образом, \[V \le P \log_2(1+S/N).\]
\end{frame}

\begin{frame}{Теорема Котельникова для цифровых сигналов}
	Так как \(F \ge 2P\), т.е. \(P \le \frac{1}{2} F\)
	
	Получаем, что скорость передачи данных (бит в секунду)
	
	\[V \le \frac{1}{2} F \log_2 (1+S/N).\]		
\end{frame}

\begin{frame}{Упражнения и вопросы}
	\begin{block}{Упражнения}
		\begin{itemize}
			
			\item
			Строго обоснуйте работу планиметра (докажите, что он вычисляет
			площадь)
			\item
			Попробуйте привести примеры «из жизни», иллюстрирующие ту же
			закономерность, которая формулируется в теореме Котельникова
		\end{itemize}
	\end{block}
	
	\begin{block}{Вопросы}
		\begin{itemize}
			
			\item
			Что такое аналоговые сигнал и устройство?
			\item
			Что такое цифровые сигнал и устройство?
			\item
			Что такое несущая? Что такое модуляция?
			\item
			Опишите известные вам виды модуляции
			\item
			Сформулируйте теорему Котельникова, объясните её смысл
			\item
			Сформулируйте теорему Шеннона
			\item
			Сформулируйте теорему Котельникова применительно к цифровым данным
		\end{itemize}
	\end{block}
\end{frame}

\postamble

\end{document}