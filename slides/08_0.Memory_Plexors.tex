\documentclass[xetex,aspectratio=43]{beamer}

\usepackage{res/lections}

\preamble

\title[Триггеры и логика]{Арифметико-логические схемы, триггеры, регистры}

\begin{document}

    \titleslide

    \tocslide

\section{Логические схемы}

\begin{frame}{Понятие логической схемы}
    \defn{Логическая схема}{устройство, вычисляющее значения одной или нескольких логических функций}
    \pause
    \begin{itemize}
        \item Как правило, всё-таки электронное
        \item \emph{Как правило}, асинхронное
        \item Не имеет \emph{состояния}, т.е. после стабилизации выходов, их значения полностью определяются входами
    \end{itemize}
\end{frame}

\subsection{Арифметико-логическое устройство}

\begin{frame}{Устройство и действие сумматора}
    \begin{itemize}
        \tightlist
        \item
        \textbf{Сумматор} (\textbf{Adder}) --- логическая схема, получающая на
        входе биты слагаемых и выдающая биты их суммы
    \end{itemize}

    Складываем \(A+B=S\).

    При суммировании двоичных чисел у \(i\)-го разряда

    \begin{itemize}
        \tightlist
        \item
        на вход поступают \(a_i, b_i, c_{i-1 \rightarrow i}\)
        \item
        на выходе имеем: \(s_i, c_{i \rightarrow i+1}\)
    \end{itemize}

    При этом:

    \begin{itemize}
        \tightlist
        \item
        \(s_i = a_i \oplus b_i \oplus c_{i-1 \rightarrow i}\)
        \item
        \(c_{i \rightarrow i+1} = (a_i \land b_i) \lor (b_i \land c_{i-1 \rightarrow i}) \lor (c_{i-1 \rightarrow i} \land a_i)\)
    \end{itemize}

    Таким образом, сумматор --- логическая схема.
\end{frame}

\begin{frame}{Схема 1-разрядного сумматора}
    \begin{block}{1-разрядный полусумматор}
        Полусумматор (2 входа)

        \begin{itemize}
            \tightlist
            \item
            \(s = a \oplus b\)
            \item
            \(c = a \wedge b\)
        \end{itemize}
    \end{block}

    \begin{block}{1-разрядный полный сумматор}
        \begin{itemize}
            \tightlist
            \item
            Оптимизируем формулу для переноса с предыдущего слайда:
        \end{itemize}

        \[c_{i \rightarrow i+1} = (a_i \land b_i) \lor (c_{i-1 \rightarrow i} \land (a_i \oplus b_i))\]

        \begin{itemize}
            \tightlist
            \item
            Это позволит создать 1-битный сумматор на основе двух полусумматоров и
            ещё 1 вентиля

            \begin{itemize}
                \tightlist
                \item
                на самом деле входы равноправны
            \end{itemize}
        \end{itemize}
    \end{block}
\end{frame}

\begin{frame}{Схема многоразрядного сумматора}
    \begin{itemize}
        \item
        Собирается через соединение переносов
        \item
        Узкое место по времени --- как раз перенос
    \end{itemize}

    \begin{block}{Замечания}
        Перенос через сумматор распространяется постепенно, поэтому:

        \begin{itemize}
            \tightlist
            \item
            в старших разряде «ответ» есть сразу, но он неправильный
            \item
            правильный ответ появляется через какое-то время
            \item
            чем больше разрядность операции, тем больше время, даже для одного и
            того же процессора
        \end{itemize}

        Последний бит переноса \(c_{N-1 \rightarrow}\) можно запомнить и при
        следующем суммировании использовать в качестве \(c_{\rightarrow 0}\).
        Это позволит соединить несколько сумматоров или несколько раз
        использовать один и тот же для реализации арифметики произвольной
        разрядности.
    \end{block}
\end{frame}

\begin{frame}{Оптимизация: упреждающий сегментированный сумматор (1)}
    \includesvg[width=\textwidth]{img/08.Lookahead_Adder.svg}
\end{frame}

\begin{frame}{Оптимизация: упреждающий сегментированный сумматор (2)}
    \begin{itemize}
        \item
        (?) --- \emph{мультиплексор} --- логическая схема, подающая на выход один из входов под действием управляющего сигнала
        \item
        Вынашивание полутора детей за 4,5 месяца. Да, оказывается можно!
        \item
        Можно сделать 3, 4 или больше сегментов
        \item
        Важна золотая середина: сегменты выбираются последовательно
        \(\Rightarrow\) грубая оценка --- порядка \(\sqrt{N}\) (на самом деле меньше)
    \end{itemize}
\end{frame}

\begin{frame}{Инвертор (для вычитания)}
    \(x-y = x+(-y)\). Проблема --- вычислить \(-y\), зная \(y\).

    \([-y] = [P-y] = [P-1+1 - y] = [1] + [(P-1) - y]\) --- уже легче,
    осталось вычислить \((P-1) - y\) (а сумматор, только прибавляющий 1,
    можно дополнительно оптимизировать)

    Для \(P = 2^N\) справедливо \(P-1 = \displaystyle\sum_{i=0}^{N-1}2^i\).

    Т.е. это число со всеми двоичными единицами. Чтобы вычесть из него
    \(y\), достаточно инвертировать все биты \(y\) при помощи вентиля НЕ,
    т.к. заёма при вычитании не возникает
\end{frame}

\begin{frame}{Умножение и деление: Мультипликатор}
    При умножении на 1 бит числа \(x\) мы можем пользоваться логической
    схемой:

    \[x\times b = x \land b.\]

    Тогда для многоразрядных чисел справедливо (\(\ll\) --- операция
    сдвига):

    \[x \times y = \sum_{i=0}^{N-1} (y_i \land (x\ll i))\]

    Т.е. выразили умножение через известные операции.

    Разрядность произведения --- сумма разрядностей множителей.
\end{frame}

\begin{frame}[fragile]{Умножение и деление: Делитель}
    Алгоритм деления «в столбик» для системы счисления с произвольным
    основанием:

    \begin{enumerate}
        \item
        сдвигаем делитель влево, пока он меньше делимого;
        \item
        вычитаем, пока вычитается, прибавляя к очередному разряду частного;
        \item
        сдвигаем делитель вправо (и меняем позицию в частном), пока нельзя
        вычитать, потом (2);
        \item
        если делитель вернулся на исходную позицию относительно сдвига, то от
        делимого остался остаток.
    \end{enumerate}

    Для двоичной:

    \begin{itemize}
        \tightlist
        \item
        вычитаем, \textbf{если} \emph{(а не пока)} вычитается.
    \end{itemize}

    \pause

    Деление медленное, многие компиляторы умеют заменять деление на
    константу умножением, пользуясь свойствами кольца, например, для 64-битных машин
    \(\mathbb{Z}/{2^{64}}\). Так,

    \mintinline{c}{a / 7}

    компилируется для x86\_64 в

    \mintinline{asm}{imul rcx, rax, 0xffffffff92492493}
\end{frame}

\begin{frame}{Умножение и деление: замечания}
        \begin{itemize}
            \item
            Мультипликатор, как и сумматор, можно реализовать логической схемой,
            но очень громоздкой, поэтому часто его делают микропрограммой. А на
            простых архитектурах --- программой.
            \item
            Есть аппаратные логические делители --- большие и «горячие»
            микросхемы, используются очень редко. Делитель --- почти всегда
            микропрограмма (или программа).
            \item
            Умножение чисел произвольной длины возможно, однако перенос
            эффективнее делать по слову (старшей половине произведения), а не по
            биту.
        \end{itemize}
\end{frame}

\subsection{Поразрядные логические операции и сдвиг}

\begin{frame}{Поразрядные логические операции}
    Очевидно
\end{frame}

\begin{frame}{Барабанная схема сдвига (Barrel shifter)}
    \begin{figure}
        \includesvg[height=0.75\textheight]{img/08.Crossbar_barrel_shifter.svg}
    \end{figure}
\end{frame}

\section{Схемы хранения состояния}

\subsection{Триггеры}

\begin{frame}{Базовые триггеры}
    \defn{Триггер}{логическая схема, запоминающая состояние}
\end{frame}

\begin{frame}{RS-триггер --- наше всё}
    \begin{enumerate}
        \tightlist
        \item
        RS-триггер
        \item
        Некорректные состояния триггера
    \end{enumerate}
\end{frame}

\begin{frame}{Синхронизация}
    \begin{enumerate}
        \tightlist
        \item
        Синхронизация по уровню тактового импульса
        \item
        Синхронизация по фронту
        \item
        D-триггер
        \item
        Связка Master-Slave
    \end{enumerate}
\end{frame}

\begin{frame}{Чередование состояний}
        \begin{enumerate}
            \tightlist
            \item
            JK-триггер
            \item
            T-триггер

            \begin{itemize}
                \tightlist
                \item
                На основе JK
                \item
                На основе D
            \end{itemize}
        \end{enumerate}

\end{frame}

\subsection{Регистры}

\begin{frame}{Регистры}
    \begin{itemize}
        \tightlist
        \item
        \defn{Регистр}{запоминающая схема на 1 или несколько битов}

        \begin{itemize}
            \item
            На практике --- с возможностью выполнения некоторых операций, т.е. с частью логики
        \end{itemize}
    \end{itemize}
\end{frame}

\begin{frame}{Счётчики}
        \begin{itemize}
            \item
            На базе T и D, по фронту и по спаду
            \item
            Увеличивающие и уменьшающие
            \item
            С заданным начальным состоянием

            \begin{itemize}
                \item
                С использованием сумматора
            \end{itemize}
        \end{itemize}
\end{frame}

\begin{frame}{Сдвигающий регистр}
    \begin{itemize}
        \item
        Загружающий
        \item
        Выгружающий

        \begin{itemize}
            \item
            Требует мультиплексора
        \end{itemize}
    \end{itemize}
\end{frame}

\begin{frame}{Хранящий регистр в процессоре}
        \begin{itemize}
            \item
            Тоже требует мультиплексора
        \end{itemize}

        \pause

        В процессоре на входе всех блоков мультиплексоры, которые выбирают,
        откуда прочитать входные данные по тактовому сигналу. Фактически
        (особенно в RISC) машинный код в необходимой последовательности
        переключает мультиплексоры, обеспечивая прохождение данных по нужным
        маршрутам.
\end{frame}

\begin{frame}{Статическая память}
    Аналогично динамической (про которую позже), но на базе мультиплексоров и регистров из D-триггеров
\end{frame}

\section*{}

\begin{frame}{Вопросы и упражнения}

    \begin{block}{Вопросы}
        \begin{itemize}
            \tightlist
            \item
            Что такое триггер?
            \item
            Что такое синхронизация по уровню, фронту и спаду тактового сигнала?
            \item
            Что такое регистр?
            \item
            Какие виды триггеров Вы знаете?
            \item
            Какие регистры Вы знаете?
            \item
            Что такое статическая память и на основе чего она создаётся
        \end{itemize}
    \end{block}

    \begin{block}{Упражнения}
        \begin{itemize}
            \tightlist
            \item
            «Нарисуйте» в Logisim или Logisim Evolution простые схемы
        \end{itemize}


    \end{block}
\end{frame}

\begin{comment}

\protect\hypertarget{slides-title}{}{Арифметические схемы, плексоры,
        память, ПЛИС}

\begin{frame}{(Де)шифраторы и плексоры}
\end{frame}

\begin{frame}
    \begin{block}{Шифратор и дешифратор}
        \begin{itemize}
            \item
            \textbf{Шифратор} (\textbf{encoder}) --- логическая схема, выдающая в
            двоичном виде номер активного (того, на котором 1) входа
            \item
            \textbf{Дешифратор} (\textbf{decoder}) --- логическая схема, выдающая
            1 на выход, заданный номером в двоичном виде
            \item
            С криптографией не связаны =)
        \end{itemize}
    \end{block}
\end{frame}

\begin{frame}
    \begin{block}{Мультиплексор и демультиплексор}
        \includesvg{images/12.mux_demux.svg}
    \end{block}
\end{frame}

\begin{frame}
    \begin{block}{Использование и неиспользование (де)мультиплексоров}
    \end{block}

    \begin{block}{Теле-\(\ldots\) связь}
        \begin{block}{Системы упаковки и свёртки/развёртки}
            \begin{itemize}
                \item
                Красным --- сигнальные линии, м.б. аналоговые, м.б. многобитные.
                \item
                В I половине XX века для временнóго мультиплексирования телефонных
                линий применялись ламповые (де)мультиплексоры.
                \item
                Не в чистом виде

                \begin{itemize}
                    \item
                    Фототелеграф (см. М.~Булгаков, Мастер и Маргарита).
                    \item
                    Современный (не в чистом виде) вариант: аналоговая телекамера
                    \(\rightarrow\) телевизор с ЭЛТ.
                \end{itemize}
            \end{itemize}

            = = = = = = =
        \end{block}
    \end{block}
\end{frame}

\begin{frame}{DRAM}
    \includesvg{images/12.dram.svg}

    Технологически конденсатор выполняется вместе с транзистором и почти не
    требует дополнительного места (может быть использована ёмкость P-N
    перехода).
\end{frame}

\begin{frame}
    \begin{block}{\href{http://en.wikipedia.org/wiki/Charlieplexing}{Чарлиплексор}}
        \includesvg{images/12.charlieplex.svg}

        \(2{\binom n 2} = 2 \mathrm{C}^n_2 = \frac{n!}{(n-2)!2!}\) комбинаций на
        \(n\) входах. Используется свойство светодиодов открывать одну самую
        короткую из параллельных цепей.
    \end{block}
\end{frame}

\begin{frame}
    \begin{block}{Flash-память: элементная база}
        \begin{itemize}
            \tightlist
            \item
            \href{http://en.wikipedia.org/wiki/Field_effect_transistor}{Униполярный
                транзистор}
        \end{itemize}

        @pause@

        \begin{itemize}
            \item
            \href{http://en.wikipedia.org/wiki/Floating-gate_transistor}{Транзистор
                с плавающим затвором} Эти транзисторы могут при подаче
            \emph{достаточно} высокого или \emph{достаточно} низкого потенциалов
            (т.е. заметно больше 1 или заметно меньше 0) на управляющий вход
            запоминать своё состояние. После нескольких миллионов срабатываний
            транзистор необратимо портится (поэтому у Flash ограничено количество
            перезаписей). Этим эффектом, так же как и широкой петлёй гистерезиса
            (хотя природа этого совершенно иная), можно пользоваться для хранения
            данных.
            \item
            На основе таких транзисторов делается
            \href{http://en.wikipedia.org/wiki/Flash_memory}{флэш-память}
        \end{itemize}
    \end{block}
\end{frame}

\begin{frame}
    \begin{block}{Flash-память: реализация NAND}
        \includesvg{images/12.nand.svg}

        \begin{itemize}
            \tightlist
            \item
            Для чтения на все слова, кроме читаемого, подаётся небольшой
            «приоткрывающий» потенциал. Ток течёт с соответствующих открытым
            транзисторам битовых линий в землю.
            \item
            Для программирования (открытия затвора, установки в бита 0) надо
            небольшим потенциалом «приоткрыть» все линии слов и подать сильный
            сигнал на пересечения нужных слова и бита.
        \end{itemize}
    \end{block}
\end{frame}

\begin{frame}[fragile]
    \begin{block}{Flash-память: особенности NAND, альтернативы}
        \begin{itemize}
            \tightlist
            \item
            Из-за особенностей изготовления транзисторов, сброс (в 1) по словам
            или ещё большим блокам
            \item
            Для SSD введена операция
            \href{https://ru.wikipedia.org/wiki/TRIM}{\texttt{TRIM}}, которая
            говорит накопителю, что блок памяти свободен и может использоваться
            для оптимизации с целью увеличения ресурса перезаписи
        \end{itemize}

        @pause@

        \begin{itemize}
            \tightlist
            \item
            Альтернативная технология ---
            \href{https://en.wikipedia.org/wiki/Phase-change_memory\#Challenges}{память
                на основе изменений фазового состояния халькогенидов} --- обладает
            потенциально лучшими характеристиками, уже используется коммерчески
            (Например, кэш для ФС Intel Optane™), но тоже со временем деградирует
        \end{itemize}

        = = = = = =
    \end{block}
\end{frame}

\begin{frame}{ПЛИС}
\end{frame}

\begin{frame}
    \begin{block}{Что это и зачем?}
        Что это?

        \begin{itemize}
            \tightlist
            \item
            \textbf{Программируемые логические интегральные схемы}
            (\textbf{Programmable Logic Device}) --- интегральные схемы,
            физический уровень которых (реализующий логику) можно задавать
            программно

            \begin{itemize}
                \tightlist
                \item
                программируются связи между компонентами схемы
            \end{itemize}
        \end{itemize}

        Зачем?

        \begin{itemize}
            \tightlist
            \item
            Применяются для эффективной реализации (быстродействие --- заметно
            медленнее серийных микросхем, существенно быстрее микропрограмм,
            принципиально быстрее обычной программной реализации) специфических
            задач

            \begin{itemize}
                \tightlist
                \item
                прототипирование, единичное или мелкосерийное производство
                \item
                микропроцессоры встроенных ЭВМ
            \end{itemize}
        \end{itemize}
    \end{block}
\end{frame}

\begin{frame}
    \begin{block}{Какие бывают ПЛИС?}
        \begin{itemize}
            \tightlist
            \item
            Простые ПЛИС --- логические функции, в т.ч. довольно сложная логика,
            но явно задаются через вентили «и», «или», «не»
            \item
            CPLD (Complex Programmable Logic Device) --- несколько более
            высокоуровневые, содержат внутренние шины, более сложные элементы
            \item
            Программируемая пользователем вентильная матрица (Field-Programmable
            Gate Array, FPGA) --- содержит готовые регистры, компоненты АЛУ и т.д.
        \end{itemize}
    \end{block}
\end{frame}

\begin{frame}
    \begin{block}{Пример: ПЛИС типа GAL (Gateway Array Logic)}
        \includesvg{images/12.PLD.svg}

        \begin{itemize}
            \tightlist
            \item
            Схема на примере вычисляет функции, представленные в ДНФ:

            \begin{itemize}
                \tightlist
                \item
                \(F_0 = B \wedge\neg C \vee A\wedge C\)
                \item
                \(F_1 = B \wedge\neg C \vee \neg A \wedge \neg B\)
                \item
                \(F_2 = A\wedge C \vee \neg A \wedge B\)
            \end{itemize}
            \item
            Входов у вентилей обычно больше
            \item
            Как реализуются перемычки в полях OR и AND?

            \begin{itemize}
                \tightlist
                \item
                «пережиганием» (одноразовое программирование)
                \item
                мультиплексором с управлением статической памятью (на основе
                D-триггера, будет позже)
                \item
                транзистором с плавающим затвором
            \end{itemize}
        \end{itemize}
    \end{block}
\end{frame}

\begin{frame}
    \begin{block}{Как программировать ПЛИС и чего можно добиться?}
        \begin{itemize}
            \tightlist
            \item
            Специальные языки, например Verilog Hardware Definition Language
            \item
            САПР (даже Logism)

            \begin{itemize}
                \tightlist
                \item
                Т.е. почти всё, что есть (было и будет) в данном курсе, можно
                сделать на ПЛИС
            \end{itemize}
            \item
            \textbf{Hardware-Software CoDesign} --- подход, подразумевающий
            совместную разработку специализированных ПО и оборудования

            \begin{itemize}
                \tightlist
                \item
                На кафедре системного программирования СПбГУ --- Булычев, Медведев,
                Терехов: https://scholar.google.com/scholar?q=HasCOL+SPbU
            \end{itemize}
        \end{itemize}

        = = = = = =
    \end{block}

    \begin{block}{Вопросы}
        \begin{itemize}
            \tightlist
            \item
            Что такое сумматор? Что такое полусумматор?
            \item
            Как реализуется полный многоразрядный сумматор?
            \item
            Как можно оптимизировать перенос в многоразрядном сумматоре?
            \item
            Как устроены вычитатели, умножители, делители?
            \item
            Что такое ширатор и дешифратор?
            \item
            Что такое мультиплексор и демультиплексор?
            \item
            На каком принципе основан и для чего используется Чарлиплексор?
            \item
            Как организована динамическая память?
            \item
            Как организована NAND-память?
            \item
            Что такое ПЛИС? Зачем они нужны и какие бывают?
            \item
            Опишите структуру ПЛИС типа GAL с полями AND и OR
            \item
            Как программируются ПЛИС?
        \end{itemize}
    \end{block}
\end{frame}



% ====================================
% ====================================

\section*{}

\begin{frame}{Вопросы}
\begin{enumerate}
\tightlist
\item
  Что такое логический вентиль?
\item
  Постройте вентиль «не» на основе реле, триодов и транзисторов
\item
  Постройте вентили «и», «или», «или-не» на основе транзисторов с
  согласующим резистором
\end{enumerate}
\end{frame}
\end{comment}

\postamble

\end{document}